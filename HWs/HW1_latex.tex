\documentclass{article}
\usepackage{ctex}
\usepackage{graphicx} % Required for inserting images
\usepackage{amsmath}
\usepackage{amssymb}
\usepackage{geometry}
\geometry{left=2cm, right=2cm, top=2cm, bottom=2cm}

\title{人工智能与机器学习基础2025-HW1}
\author{TA: 郑悟强}
\date{September 2025}

\begin{document}

\maketitle
\section{模型偏差 (Model Bias)}
在进行机器学习任务时,我们有时候模型最终的表现可能并不能达到非常理想的程度。除去优化方法等限制外,还有可能是我们的模型本身存在能力的上限。在这个问题中,我们会一起分析一下,从模型角度来看,为什么误差会存在,误差的上限在哪里?进一步,我们可以从结果分析一下如何尽量让这个误差变小。

\noindent\textbf{(a)}
对于我们的训练数据集 $\mathcal{D}_{train}\{(x_i, y_i)\}_{i=1}^N$,存在其最优拟合函数$y = f_{\text{true}}(x) + \epsilon$,这里的误差 $\epsilon$服从期望为0,方差为 $\sigma^2$的分布。实际上,我们训练时可能会随机选取到整个训练集的一个子集,拟合出来的模型为 $f_{\hat{w}(\text{train})}$。进一步,我们假设我们一共这样选取了 $n$次训练子集,得到的所有的模型的均值为 $\bar{f}$,那么,事实上,我们理论上的泛化误差,用MSE作为度量方式,可以由此推导:
\begin{equation}
    \mathbb{E}_{\text{train}}[\text{{Generalization Error}}] = \mathbb{E}_{\text{train}}[\mathbb{E}_{x, y}[(y - f_{\hat{w}(\text{train})}(x))^2]] = \sigma^2 + \text{Bias}^2 + \text{Var}(f),
\end{equation}
其中:
$$
\text{Bias}^2 = \mathbb{E}_{\text{train}}[(\bar{f} - f_{\text{true}})^2], \quad \text{Var}(f) = \mathbb{E}_{\text{train}}[(\bar{f} - f_{\hat{w}(\text{train})})^2]
$$
问题:请你补全这个推导过程

\noindent\textbf{(b)}
请说明,与训练数据无关,模型偏差 $Bias^2$ 非负且总是存在。并且回答,什么时候,模型偏差永远大于0呢?(可以举一个栗子)

\noindent \textbf{(c)}
请说明,$Bias^2$和Var分别与 $N$相关还是与 $n$相关,成什么关系?($\mathcal{O}(?)$)。具体的,在神经网络中,模型的复杂度与这两项的关系是什么样的?

\section{数据偏差 (Data Bias)}
除了上题中讲述的模型能力的偏差,实际上,可能训练数据本身也限制了你的模型的表现。主要原因在于,我们的有限样本 $S$ 无法准确泛化到整体的期望数据分布 $D$,我们对这种有限数据带来的泛化限制称作数据偏差。

\noindent \textbf{(a)}
假设我们在训练集训练得到的模型为 $h \in H$,其中 $H$为整个模型代表的集合,比如,$H$为所有的线性模型,$h$为其中我们训练得到的一个固定参数的线性模型。我们定义这样的误差函数:
\begin{equation}
    \text{err}_D(h) = \mathbb{E}_{(x, y) \sim D}[I(h(x) \neq y)], \quad \text{err}_S(h) = \frac{1}{n} \sum_{i=1}^n I(h(x_i) \neq y_i), 
\end{equation}
其中 $I(\cdot, \cdot)$为示性函数(相等为1,不等为0)。那么,我们可以这样定义这个 $h$的泛化性是不好的:
\begin{equation}
    |\text{err}_D(h) - \text{err}_S(h)| \geq \epsilon, 
\end{equation}
其中 $\epsilon$是我们定义的一个误差界。

请证明:对于固定的 $h$,$\Pr [|\text{err}_D(h) - \text{err}_S(h)| > \epsilon] \leq 2 \exp(-2n\epsilon^2)$

提示:可以使用Hoeffding不等式:对于独立有界随机变量 \( Z_i \in [a,b] \),\( \Pr[|(1/n)\sum Z_i - \mathbb{E}| > \epsilon] \leq 2 \exp(-2n\epsilon^2 / (b-a)^2) \)。

\noindent \textbf{(b)}
现在,我们给这个结论泛化一下,用 $|H|$表示整个模型类的空间大小。

请证明:$\Pr [\exists h \in H, |\text{err}_D(h) - \text{err}_S(h)| > \epsilon] \leq 2|H| \exp(-2n\epsilon^2)$

\noindent \textbf{(c)}
请尝试解释一下这个结论的直观意义,我们的数据偏差与哪些因素有关?呈什么样的相关性变化趋势?我们有哪些方法来尽量减缓数据偏差带来的影响?

\section{贝叶斯分类器}
在课堂上,老师讲述了一种线性的分类器,逻辑回归。从数学上,其基本原理在于,直接建模 $p(y | x)$这个概率值,其中$x$是样本的特征,$y$是类别。逻辑回归通过直接优化这个概率,通过梯度下降的更新策略,从数据中学到这个概率的建模。

但这种做法也存在一定的问题,比如训练成本的高昂,我们需要从数据集中不断采样更新模型,并且优化方式的限制导致并不一定能轻松训练得到非常好的模型。于是我们想,既然直接建模 $p(y | x)$很难,那能不能做一定的数学上的变化,得到更简单优美的方式呢?

考虑二分类任务,$y_1, y_2$表示样本属于类别1或类别2,我们对这个概率做一个贝叶斯公式的展开:
\begin{equation}
    p(y_1 | x) = \frac{p(x|y_1) p (y_1)}{p(x|y_1)p(y_1) + p(x|y_2)p(y_2)}.
\end{equation}
我们可以做这样的假设,我们假设每个类别的数据特征服从各自的高斯分布,两个高斯分布的期望不同,但方差相同(可以想想这么简化有什么意义?)
\begin{equation}
    f_{\mu, \Sigma}(x) = \frac{1}{\sqrt{(2\pi)^Ddet(\Sigma)}}exp\{-\frac{1}{2}(x - \mu)^T \Sigma^{-1}(x - \mu)\}.
\end{equation}
其中,$D$是特征$x$的维度,类别1的样本期望为$\mu^1$,方差为$\Sigma$,数据集中,类别1有$n_1$个样本。类别2的样本数据期望为$\mu^2$,方差为$\Sigma$,数据集中有$n_2$样本。即 $x | y_1 \sim \mathcal{N}(\mu^1, \Sigma), x | y_2 \sim \mathcal{N}(\mu^2, \Sigma)$

\noindent \textbf{(a)}
请通过极大似然估计的方法,推导出$\mu^1, \mu^2, \Sigma$的显式解。提示:结果中应该只包含 $n_1, n_2$, $\{x_i^1\}_{i=1}^{n_1}$(类别1的所有数据特征), $\{x_i^2\}_{i=1}^{n_2}$(类别2的所有数据的特征)。

提示:我们希望优化 $\mathcal{L}(\mu^1, \mu^2, \Sigma) = \prod_{i=1}^{n_1} p(x_i^1|y_1) \prod_{i=1}^{n_2} p(x_i^2|y_2)$ 

\noindent \textbf{(b)}
表~\ref{table:testdata}中,我们提供了一个简单的三特征的测试数据:
\begin{table}[!h]
    \centering
    \begin{tabular}{cccc}
        \hline
        \( x_1 \) & \( x_2 \) & \( x_3 \) & 标签 \\
        \hline
        2.0 & 2.5 & 2.0 & 0 \\
        2.5 & 2.8 & 2.2 & 0 \\
        3.0 & 2.7 & 2.5 & 0 \\
        2.2 & 3.0 & 2.3 & 0 \\
        2.8 & 2.6 & 2.4 & 0 \\
        3.5 & 3.8 & 3.2 & 1 \\
        3.2 & 4.0 & 3.5 & 1 \\
        3.8 & 3.5 & 3.7 & 1 \\
        3.0 & 3.9 & 3.3 & 1 \\
        4.0 & 3.6 & 3.9 & 1 \\
        \hline
    \end{tabular}
    \caption{测试数据(10条样本,3个特征,二分类)}
    \label{table:testdata}
\end{table}

用贝叶斯分类器建模后,两个类别的期望分别是多少?方差是多少?
现在我们有1个新的数据,请帮我判断它应该属于哪个类别,概率是多少?新数据:$(2.7, 2.9, 3.5)$

\noindent \textbf{(c)}
事实上,贝叶斯分类器也可以写成一种逻辑回归,我们可以这样处理:
\begin{equation}
    p(y_1 | x) = \frac{p(x|y_1) p (y_1)}{p(x|y_1)p(y_1) + p(x|y_2)p(y_2)} = \frac{1}{1 + \frac{p(x|y_2)p(y_2)}{p(x|y_1)p(y_1)}} = \sigma(z),
\end{equation}
我们可以给下面那一项看作 $z = \ln\frac{p(x|y_2)p(y_2)}{p(x|y_1)p(y_1)}$。

事实上,这个$z$相对于$x$也是线性的,即我们可以写成$z = w \cdot x + b$。请尝试用$\mu^1, \mu^2, \Sigma, n_1, n_2$推导出$w, b$。

Comments:\textit{从这里我们能够看到,其实逻辑回归与贝叶斯分类都是线性分类器,不过区别在于,逻辑回归通过优化目标的梯度下降进行学习,贝叶斯分类器采用直接通过高斯分布的假设寻找$w, b$的显式解。二者互有优劣。比如逻辑回归表达能力强,往往可以实现更好的泛化性,但存在优化困难等问题。贝叶斯分类器优化简单,只需要过一遍数据计算对应的参数即可,但存在表达能力弱,且理论最优解会导致容易过拟合,泛化性差。你也可以从另一种角度来理解这个问题。我们知道,线性回归是存在理论最优解的,但逻辑回归不行,直接求梯度后会发现,这个等式不存在解析解。所以我们可以对数据加上随机性(高斯分布)的假设,通过一定的数学方法求出其最优解。这样会带来更强的假设,但会得到一个更加优美的数值解。}

\section{神经网络}

考虑下图所示的神经网络,其中所有隐藏神经元都使用ReLU 激活函数(图中的f 1),输出层使用softmax 激活函数(图中的$f^2$),输出为softmax 输出(图中的$a_1^2$ 和$a_2^2$)。

\includegraphics[width=0.95\textwidth]{nn.png}

给定输入$x = [x_1, x_2]^T$,网络的隐藏单元按以下方程分阶段激活:

$$
\begin{array}{ll}
z_1^1=x_1 w_{1,1}^1+x_2 w_{2,1}^1+w_{0,1}^1 & a_1^1=\max \left\{z_1^1, 0\right\} \\
z_2^1=x_1 w_{1,2}^1+x_2 w_{2,2}^1+w_{0,2}^1 & a_2^1=\max \left\{z_2^1, 0\right\} \\
z_3^1=x_1 w_{1,3}^1+x_2 w_{2,3}^1+w_{0,3}^1 & a_3^1=\max \left\{z_3^1, 0\right\} \\
z_4^1=x_1 w_{1,4}^1+x_2 w_{2,4}^1+w_{0,4}^1 & a_4^1=\max \left\{z_4^1, 0\right\}
\end{array}
$$

$$
\begin{aligned}
& z_1^2=a_1^1 w_{1,1}^2+a_2^1 w_{2,1}^2+a_3^1 w_{3,1}^2+a_4^1 w_{4,1}^2+w_{0,1}^2 \\
& z_2^2=a_1^1 w_{1,2}^2+a_2^1 w_{2,2}^2+a_3^1 w_{3,2}^2+a_4^1 w_{4,2}^2+w_{0,2}^2
\end{aligned}
$$

网络的最终输出通过对最后一层应用softmax 函数得到:

$$
\begin{aligned}
& a_1^2=\frac{e^{z_1^2}}{e^{z_1^2}+e^{z_2^2}} \\
& a_2^2=\frac{e^{z_2^2}}{e^{z_1^2}+e^{z_2^2}}
\end{aligned}
$$

在这个问题中,我们将考虑以下参数设置:

$$
\begin{aligned}
{\left[\begin{array}{cccc}
w_{1,1}^1 & w_{1,2}^1 & w_{1,3}^1 & w_{1,4}^1 \\
w_{2,1}^1 & w_{2,2}^1 & w_{2,3} & w_{2,4}^1 \\

\end{array}\right]=\left[\begin{array}{cccc}
1 & 0 & -1 & 0 \\
0 & 1 & 0 & -1
\end{array}\right], } & {\left[\begin{array}{l}
w_{0,1}^1 \\
w_{0,2}^1 \\
w_{0,3}^1 \\
w_{0,4}^1
\end{array}\right]=\left[\begin{array}{l}
-1 \\
-1 \\
-1 \\
-1
\end{array}\right], } \\
{\left[\begin{array}{cc}
w_{1,1}^2 & w_{1,2}^2 \\
w_{2,1}^2 & w_{2,2}^2 \\
w_{3,1}^2 & w_{3,2}^2 \\
w_{4,1}^2 & w_{4,2}^2
\end{array}\right]=\left[\begin{array}{cc}
1 & -1 \\
1 & -1 \\
1 & -1 \\
1 & -1
\end{array}\right], } & {\left[\begin{array}{l}
w_{0,1}^2 \\
w_{0,2}^2
\end{array}\right]=\left[\begin{array}{l}
0 \\
2
\end{array}\right] }
\end{aligned}
$$

\noindent\textbf{(a)}
考虑输入$x_1 = 3, x_2 = 14$。网络隐藏单元的输出$\left(f^1\left(z_1^1\right), f^1\left(z_2^1\right), f^1\left(z_3^1\right), f^1\left(z_4^1\right)\right)$ 和最终输出 $\left(a_1^2, a_2^2\right)$ 是什么?

\noindent\textbf{(b)}
考虑以下输入向量:$x^{(1)}=[0.5,0.5]^T, \quad x^{(2)}=[0,2]^T, \quad x^{(3)}=[-3,0.5]^T$。输入一个矩阵,其中每一列表示每个输入向量的隐藏单元的输出$\left(f\left(z_1^1\right), \ldots, f\left(z_4^1\right)\right)$。

\noindent\textbf{(c)}
使用交叉熵损失函数(Cross-Entropy Loss),给定输入样本$(x_1 = 3, x_2 = 14)$ 和目标向量 $(y_1, y_2) = (0, 1)$,执行一次反向传播步骤,以学习率$\eta = 0.1$ 更新网络中的每一个权重。

\end{document}
